\subsection{Premières \routes{} \& composants React}

\subsubsection[welcome!]{welcome!\label{sec:welcome!}}
En \laravel{} + \inertia{}, les \routes{} se trouvent toujours dans \verb|routes/web.php|.  
Une \route{}, c’est simplement un chemin que Laravel surveille. Quand quelqu’un visite ce chemin dans le navigateur, Laravel sait quoi afficher.

Voici la route par défaut que vous trouverez dans un projet neuf :
\begin{figure}[H]
    \centering
    \includegraphics[width=0.75\textwidth]{figures-C1/basic_route.png}
\end{figure}

\begin{itemize}
    \item \verb|Route::get('/', ...)| signifie : « Quand quelqu’un va à l’adresse \verb|/| (la page d’accueil), fais ce qu’il y a dans les accolades ».
    \item Ici, on utilise \verb|Inertia::render('Welcome')| pour dire à Laravel : « Affiche la page React \texttt{Welcome} ».
    \item \texttt{Welcome} correspond à un fichier \verb|Welcome.jsx| qui se trouve dans \verb|resources/js/Pages|.
\end{itemize}

\subsubsection[Welcome]{Welcome}

\begin{figure}[H]
  \centering
  \begin{minipage}[t]{0.50\textwidth}
    \vspace{0pt}\raggedright
    Allons voir le fichier \verb|Welcome.jsx|. \\
    Par défaut, il contient beaucoup de code inutile pour notre formation : styles, textes et boutons de démonstration.  
    On va tout effacer et repartir d’une base simple.\\
    Voici à quoi ressemble le fichier \verb|Welcome.jsx| après nettoyage.
    Il ne contient qu’un composant React minimal, prêt à accueillir notre contenu.
  \end{minipage}\hfill
  \begin{minipage}[t]{0.40\textwidth} % réduit de 0.46 à 0.40
    \vspace{0pt}\centering
    \includegraphics[width=0.9\linewidth]{figures-C1/welcome_jsx_empty.png} % 0.9 au lieu de 1.0
  \end{minipage}
\end{figure}

\paragraph{Explication du code}
\begin{itemize}
    \item \verb|export default function Welcome()| : on crée un composant React qui s’appelle \textbf{Welcome} et on l’exporte pour qu’il puisse être utilisé ailleurs.
    \item \verb|<div style={{ ... }}>| : on crée une boîte (\texttt{div}) et on lui applique du style directement en JavaScript grâce à la syntaxe \verb|{{ ... }}|.
    \item \verb|display: "flex"| : active \textbf{Flexbox}, une méthode pratique pour centrer des éléments.
    \item \verb|justifyContent: "center"| : centre le contenu horizontalement.
    \item \verb|alignItems: "center"| : centre le contenu verticalement.
    \item \verb|height: "100vh"| : la boîte prend toute la hauteur de l’écran (\texttt{vh} = \textit{viewport height}).
    \item \verb|fontSize: "2rem"| : définit la taille du texte à deux fois la taille normale.
    \item \verb|fontWeight: "bold"| : met le texte en gras.
\end{itemize}

\begin{center}
    \fbox{\includegraphics[width=0.65\textwidth]{figures-C1/welcome_page.png}}
\end{center}

Résultat : une page toute blanche avec juste \textbf{Welcome} bien centré, qui servira de point de départ pour construire la suite du site.

\paragraph{À retenir :}
\begin{enumerate}
    \item Les \routes{} définissent quel composant React sera affiché.
    \item \verb|Inertia::render('Welcome')| charge le fichier \verb|Welcome.jsx|.
    \item On part volontairement d’une page simple et vide pour mieux comprendre ce que l’on ajoute ensuite.
\end{enumerate}

\subsubsection[Controller][laravel.com/docs/10.x/controllers\#introduction]{Controller}

Bon, il est temps de remplir tout ca.

\begin{wrapfigure}[9]{r}{0.25\textwidth}
    \vspace{-0.5cm}
    \includegraphics[width=0.25\textwidth]{figures-C1/3_premieres_views.pdf}
\end{wrapfigure}

Commençons par créer un fichier \verb|pages| dans \verb|resources/views/|, et ajoutez trois autres fichiers comme indiqué ci-contre.

Ensuite, il va falloir créer un \controller{}. Pour cela, tapez
\verb|sail artisan make:controller PagesController|\footnote{\verb|sail artisan make:| est une commande très utile pour créer énormément de fichiers que nous verrons plus tard. Comme nous l'utilisons au travers de \laravelsail, il faut utiliser \verb|sail artisan| à la place}. \linebreak Les \controllers{} se trouvent dans \verb|app\Http\controllers\|. Dans \verb|PagesController|, créez trois fonctions comme à la \textsc{Figure }\ref{fig:PagesController1}.

\SaveVerb{term}|PagesController|
\begin{figure}[!h]
    \centering
    \begin{subfigure}{0.49\textwidth}
        \centering
        \includegraphics[width=\textwidth]{figures-C1/pages_controller_1a.pdf}
    \end{subfigure}
    \begin{subfigure}{0.49\textwidth}
        \centering
        \includegraphics[width=\textwidth]{figures-C1/pages_controller_1b.pdf}
    \end{subfigure}
    \caption{\protect\UseVerb{term}\label{fig:PagesController1}}
\end{figure}

Vous l'aurez compris, \verb|return view()| permet d'afficher la \view{} donnée en argument: en l'occurence, les 3 \views{} \verb|index|, \verb|services| et \verb|about|. Notez que \verb|pages.| indique que ces 3 \views{} se trouvent dans le dossier \verb|pages|.

\newpage 

Si vous vous rappellez bien de ce qu'on a vu plus tôt, chaque \view{} \html{} doit contenir un tag \verb|<head>|. Celui-ci sera le même pour chaque page donc il serait judicieux\footnote{\textbf{D.R.Y}: \textit{Don't Repeat Yourself!}} de créer une sorte de template dans lequel on mettrait le \verb|<head>| et qui sera ensuite utilisé pour les 26854 pages que comptera bientôt notre site! 

COMME PAR HASARD les fichiers \verb|.blade.php| nous permettent de faire cela: commencez par créer un dossier \verb|layouts| dans les views, et créez un fichier dedans appellé \verb|app.blade.php|. Ensuite, remplissez-le avec le contenu de \verb|welcome.blade.php| (que vous pouvez désormais supprimer), puis ajoutez la commande \verb|@yield('content')| dans le \verb|<body>|. Enfin, il ne reste plus qu'a utiliser ce layout pour remplir vos 3 nouvelles \views{}.

\SaveVerb{about}|about|
\SaveVerb{services}|services|
\SaveVerb{index}|index|
\begin{figure}[!h]
    \centering
    \begin{subfigure}[b]{0.49\textwidth}
         \centering
         \includegraphics[width=\textwidth]{figures-C1/basic_about.pdf}
     \end{subfigure}
     \begin{subfigure}[b]{0.49\textwidth}
         \centering
         \includegraphics[width=\textwidth]{figures-C1/basic_services.pdf}
     \end{subfigure}
     \begin{subfigure}[b]{1\textwidth}
         \centering
         \includegraphics[width=\textwidth]{figures-C1/basic_accueil.pdf}
     \end{subfigure}
        \caption{Contenu des \views{} \protect\UseVerb{about} (à gauche), \protect\UseVerb{services} (à droite) et \protect\UseVerb{index} (en bas)}
\end{figure}
Petit tuto \html{} rapide: le tag \verb|<p>| renferme un pparagraphe, et le tag \verb|<h1>| contient lui un h1titre. De même, \verb|<h2>| désignera un h2sous-titre, \verb|<h3>| un h3sous-sous-titre, \verb|<h4>| un h4sous-sous-sous-titre, \ldots

Que se passe-t'il exactement? Chacune des \views{} va prendre le contenu du layout \verb|app| (via \verb|@extends()|), et remplir sa section \verb|content| par ce qu'il y a entre \verb|@section| et \verb|@endsection|. Simple et efficace! Nous rajouterons d'autres choses dans ce layout par la suite.

\newpage

Enfin, pour pouvoir admirer le fruit de votre dur labeur, il faut créer les \routes{} qui permettront d'afficher ces pages. Pour cela, rendez-vous dans \verb|web.php|:

\begin{wrapfigure}[16]{r}{0.7\textwidth}
    \vspace{-0.5cm}
    \includegraphics[width=0.7\textwidth]{figures-C1/3_premieres_routes.pdf}
    \caption{}
\end{wrapfigure}
Afin d'utiliser notre nouveau controller, il faut le déclarer pour que \laravel{} sache qu'il existe. C'est à ca que sert ``\verb|use gngngn|'', (ce qui est suit est plutôt évident).
Ensuite, décortiquons ce qui se passe: Comme précédement, le 1er argument de \verb|get()| donne l'adresse. Par exemple, la 2eme route est appellée à l'URL \url{http://tutorialstepbystep/services}. Le deuxième argument donne dans une \texttt{array} le \controller{} ainsi que sa méthode à exécuter. Pour la deuxième route, aller à l'URL mensionnée va donc exécuter la fonction \verb|services()| que nous avons créée il y a 5 (ou 40) minutes. Celle ci nous retourne la \view{} correspondante, donc en allant sur cet URL nous voyons dans un coin de l'écran:

\begin{figure}[!h]
    \centering
    \fbox{\includegraphics[]{figures-C1/basic_services_web.pdf}}
    \caption{page internet moche}
\end{figure}

C'est laid, pas vrai? Nous allons améliorer cela à la prochaine section.

\newpage
