\subsection{Conclusion}

\subsubsection[Postambule]{Postambule}

Ce tutoriel touche (enfin) à sa fin! Ce fut long et sûrement douloureux pour vous, mais vous y êtes arrivés, ce qui mérite déjà des félicitations en soi. Donc, bravo!

Si vous avez toujours l'impression d'être un peu perdu, c'est tout à fait normal. Nous avons vu beaucoup de choses différentes, plusieurs langages différents, et des mécaniques qui s'entrecroisent. Ce n'est pas en suivant un tutoriel que vous deviendrez les maîtres de la programmation, et ce n'est pas l'objectif de ce tutoriel. Son objectif est de vous familiariser avec différentes notions afin de vous donner les clés nécessaires pour continuer de vous perfectionner et de comprendre les choses par vous-mêmes. J'espère avoir accompli cet objectif!

\subsubsection[Et ensuite?]{Et ensuite?}

Il y a plusieurs suites possibles pour vous améliorer. 

Premièrement, vous pouvez lire les documentations données au fil du tutoriel afin d'assimiler les notions et d'aller plus loin. Vous pouvez vous-mêmes essayer d'ajouter des fonctionnalités à ce mini-site (ou créer le vôtre de A à Z) pour tester vos nouvelles compétences.

Ou alors, si vous continuez l'aventure \nhitec{} avec nous, vous aurez accès à encore plus de ressources pour continuer de vous former et d'apprendre des choses avec nous, et si vous le souhaitez, vous pourrez même ajouter votre pierre à l'édifice en contribuant à cette formation ou en travaillant sur les nombreux projets que nous avons à proposer!

\nhitec{} vous attend!