\subsection{React, c'est quoi ?}

\subsubsection[Prélude]{Prélude}

\react{} est ce qu’on appelle une \textit{bibliothèque \js{}}. Contrairement à un framework comme \laravel{}{}, React ne cherche pas à tout contrôler: son rôle principal, c’est de vous aider à créer des \textbf{interfaces utilisateur} (les pages web que vous voyez à l’écran). 

Dit autrement: si \laravel{} s’occupe des coulisses (\textit{backend}), React, lui, gère ce que l’utilisateur voit et avec quoi il interagit (\textit{frontend}).\footnote{Vous pouvez consulter la documentation officielle de React à l'adresse suivante : \href{https://react.dev}{https://react.dev}} 

\underline{Astuce:} Pensez à \react{} comme à une boîte de Lego: chaque brique (un \texttt{composant}) peut être réutilisée pour construire des interfaces plus grandes et plus complexes.

\subsubsection[Fonctionnement et philosophie]{Fonctionnement \& philosophie}

React repose sur quelques concepts clés:

\begin{enumerate}
    \item \texttt{Composants:} Chaque partie de votre site (bouton, barre de navigation, formulaire, etc.) est un composant. Ces composants peuvent être combinés pour former une page complète.
    \item \texttt{JSX:} \react{} utilise une syntaxe qui mélange \js{} et \html{}. Cela peut sembler étrange au début, mais c’est ce qui rend \react{} puissant et lisible.
    \item \texttt{Props:} Les composants peuvent recevoir des informations de l’extérieur (comme des paramètres dans une fonction). On appelle ça des \texttt{props}.
    \item \texttt{État (State):} Un composant peut garder en mémoire des informations qui changent au fil du temps (par exemple: le contenu d’un champ de recherche). C’est ce qu’on appelle l’état.
    \item \texttt{Virtual DOM:} \react{} ne modifie pas directement la page, mais une copie virtuelle de celle-ci. Quand quelque chose change, il met à jour uniquement la partie nécessaire, ce qui rend l’application rapide.
\end{enumerate}

\subsubsection[Pourquoi l'utiliser ?]{Pourquoi l'utiliser ?}

Alors pourquoi se compliquer la vie avec \react{}?

\begin{itemize}
    \item Parce qu’il permet de créer des interfaces modernes, interactives et rapides.
    \item Parce qu’il est utilisé partout (Facebook, Instagram, Airbnb, Netflix...).
    \item Parce qu’il est \textbf{composantiel}: vous écrivez une fois un bouton, et vous pouvez le réutiliser 20 fois dans l’application sans copier-coller.
    \item Parce qu’il a une \textbf{énorme communauté}: dès que vous avez un problème, il y a probablement déjà une solution sur Internet.
\end{itemize}

Bref, React, c’est un peu comme passer d’un vieux Nokia 3310 à un smartphone: ça ouvre un nouveau monde de possibilités. 
\vsp

Maintenant que nous avons \laravel{} pour gérer le \textbf{backend} et \react{} pour construire le \textbf{frontend}, 
il reste une question cruciale : \textit{comment faire pour que ces deux mondes communiquent efficacement entre eux ?}  
C’est précisément là qu’intervient \inertia{} .  
Dans la prochaine section, nous allons décortiquer ensemble comment Inertia nous permet de marier React et Laravel 
sans avoir besoin de construire une API séparée.